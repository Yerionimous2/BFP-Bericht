\documentclass[12pt]{article}
% Für spezialisierte Befehle etc.
% config.tex
\usepackage[
  a4paper,
  left=2.5cm,
  right=2.5cm,
  top=2.5cm,
  bottom=2cm
]{geometry}

\usepackage{tcolorbox}
\usepackage{amssymb}
\usepackage{amsmath}
\usepackage[ngerman]{babel}
\usepackage[T1]{fontenc}
\usepackage{pdfpages}
\usepackage[utf8]{inputenc}
\usepackage{tikz}
\usepackage{longtable}
\usepackage{array}
\usepackage{mathabx}
\usepackage{todonotes}
\usepackage[hidelinks]{hyperref}
\usepackage{setspace}
\usepackage{mathptmx}
\onehalfspacing
\usetikzlibrary{positioning, shapes.geometric, shapes.multipart, arrows, calc}
\tcbuselibrary{listings, skins, breakable}

\newcounter{block}[section]
\renewcommand{\theblock}{\thesection.\arabic{block}}

\usepackage{etoolbox}
\preto\section{\setcounter{block}{0}}

%Alle Sections beginnen auf einer neuen Seite
\let\oldsection\section
\renewcommand{\section}{\clearpage\oldsection}

\newcommand{\insertpdf}[2][0.8]{%
  \includepdf[
    pages=-,
    scale=#1,
    pagecommand={%
      \thispagestyle{plain}%
      \vspace*{0.5cm}%
      \centering\small\texttt{#2}%
    }
  ]{#2}%
}


\newtcolorbox{block}[1]{
  enhanced,
  breakable,
  colback=green!5!white,
  colframe=green!65!black,
  coltitle=black,
  fonttitle=\bfseries,
  title=\thesection.\number\numexpr\value{block}+1\relax\ #1,
  before upper={},
  after upper={\stepcounter{block}},
  boxrule=0.8pt,
  arc=0mm,
  left=1mm,
  right=1mm,
  top=1mm,
  bottom=1mm,
}

\newenvironment{beweis}[1][Beweis]{
  \par\noindent\textbf{#1.}\\ 
}{
    \hfill \framebox[1.5ex][c]{}
    \par\vspace{2em}
    \noindent
}

\newcolumntype{P}[1]{>{\centering\arraybackslash}p{#1}}

% Abgabe in 6 Wochen(vom 19.1. an, das ist der 2. März)

% Mail mit Betreff: Peter Minor, 514507, BFP-Seminar WiSe 25/26
% Gibt wohl keine Eingangsbestätigung

\begin{document}

% Titelseite
\thispagestyle{empty}
\noindent
Universität Münster \\
Fachbereich 06, Erziehungswissenschaft und Sozialwissenschaften \\
Wintersemester 2025/2026 \\
Vorbereitung und Begleitung des Berufsfeldpraktikums \\
Janusz Wilden \\
063591 \\

\vspace{2cm}

\begin{center}
\LARGE \textbf{Theoriebasierte Praxisreflexion zum Berufsfeldpraktikum}
\date{}
\end{center}

\vspace*{\fill}
\begin{flushright}
Peter Minor \\
514507\\
Boeselagerstraße 75b \\
48163 Münster \\
pminor@uni-muenster.de\\
Zweifach-Bachelor \\
Mathematik, Informatik \\
11. Fachsemester \\
Universität Münster \\
1. April 2025 - 30. September 2025\\
Abgabedatum \today \\
\end{flushright}
\newpage

\listoftodos % Vor Abgabe entfernen

% Inhaltsverzeichnis
\thispagestyle{empty}
\tableofcontents
\newpage


\section{Einleitung: Beschreibung der Praktikumstätigkeit}
Ich habe mein Praktikum an der Universität Münster als Tutor für die Vorlesung Informatik II absolviert,
die sich hauptsächlich an Studierende der Informatik, sowohl Hauptfachlich, als auch als Fach beim Zweifachbachelor
Informatik und an Wirtschaftsinformatikstudierende richtet. Sie behandelt weiterführend die
Grundkonzepte des objektorientierten Programierens und der Softwareentwicklung und -planung, insbesondere Rekursion,
Datenstrukturen und Komplexitätsanalyse von Algorithmen.

Ich habe einen Großteil meiner Tätigkeit im Homeoffice bzw. digital ausgeführt, die Zeit, die ich auf
dem Campus an der Einsteinstraße 64 verbrachte, hatte ich einen mir zugewiesenen Raum im Seminarraumzentrum am Orléans-Ring 12, den ich für die Zeit meines Tutoriums
frei nutzen durfte.

Mein Tätigkeitsbereich umfasste das Korrigieren von Abgaben der Übungsblätter, die die Studierenden im
Zuge der Vorbereitung auf die Klausur und im Rahmen einer Studienleistung einreichen mussten. Dazu waren
die Studierenden in Gruppen von 2 oder 3 Studierenden eingeteilt. Diese Abgaben erfolgten wöchentlich,
für die Korrektur war sowohl eine Musterlösung als auch ein Bewertungsschema vorgegeben. Außerdem zählte
das Halten eines Tutoriums zu meinem Aufgabenbereich, in dem ich die Vorlesungsinhalte der vergangengen
Woche und jeweils die Korrektur des vorangegangenen Übungsblattes und eine Vorbesprechung des aktuell zu
bearbeitenden Übungsblattes behandelte. Dabei sollte ich die Vorlesungsinhalte auf eine neue Art und an die jeweiligen
Kenntnisse der Studierenden, die ich aus der Qualität der Abgaben herauslesen sollte, angepasst darstellen und
Feedback zu den Abgaben geben, sowie die Probleme, die sich durch die Lernstandskontrollen in Form der Abgaben
gezeigt haben, ansprechen und bei der Lösungsfindung unterstützen. Dieses Feedback sollte zusätzlich zu der
Planung und Durchführung des Tutoriums auch einen Einfluss auf die Korrektur der Abgabe haben, was in Form von
Korrekturkommentaren und Bepunktungen erfolgen sollte.
Neben den Vorgegangenen, welche meine Hauptaufgaben darstellten, sollte ich den Studierenden auch organisatorisch
und beratend zur Seite stehen. Ich hatte eine feste Gruppe aus 16 Studierenden, die zu meiner Übungsgruppe
gehörten, also zu meinem Tutorium kamen, mich als ersten Ansprechpartner hatten und deren Übungszettelabgaben
ich beurteilte. Der letzte Teil meiner Arbeitszeit bestand aus einer Übungsleitendenbesprechung, in der alle
Tutoren sich versammelten und der Professor, der die Vorlesung hielt, uns über das aktuelle Geschehen in der
Vorlesung und die jeweils aktuellen Übungsblätter informierte. Ein großer Teil dieser Tätigkeiten ist
Eigenständig erfolgt.

Bis auf die Übungsleitendenbesprechung hatte ich keinen professionellen Kontakt zu meinen Mitübungsleitenden
oder dem Professorenteam. Meine Studierenden hatten die Möglichkeit, mich per Mail, Learnweb oder mithilfe
einer eigens für eine unkomplizierte Kommunikation eingerichtete Whatsapp-Gruppe zu kontaktieren. 

\section{Beschreibung des Fallbeispiels}
\subsection{Verhältnis zu E.}
In meiner Übungsgruppe gab es eine Studierende Person E., mit der ich bereits persönlich Kontakt hatte. Ich
ging regelmäßig mit ihr in die Mensa zum Essen oder traf sie im Kontext der gemeinsamen Freundesgruppe.
Unser freundschaftlicher Kontakt und die Begegnungen im universitären Kontext waren nur dadurch verschränkt, dass wir
im freundschaftlichen Kontext über die Vorlesung sprachen und sie mir Feedback zu meiner Tutoriumsgestaltung gab. Anders
herum war der einzige Einfluss, dass ich ihren Namen bereits kannte und wir nach den Tutorien gemeinsam essen
gingen oder kurz abglichen, ob wir zu einer Verabredung innerhalb der Freundesgruppe gehen würden.

\subsection{Abgabegruppe}
Am Anfang des Semesters werden die jeweiligen Abgabegruppen eingeteilt, in denen die Übungsblätter zusammen bearbeitet
und abgegeben werden. E. hat sich in der von mir dafür vorgesehenen Kennlernphase einer Gruppe zweier befreundeter
Studierenden, Person T. und Person D., angeschlossen. Ich anonymisierte die Abgaben, für mein persönliches
Fairnessempfinden, also wusste ich nicht, welche Personen welche Abgabe machten. Mir ist eine Abgabegruppe aufgefallen,
die immer diesselbe Handschrift hatte, was ich mir dadurch erklärte, dass eine Person der Abgabegruppe die Gemeinsamen
Lösungen aufschrieb, um eine bessere Lesbarkeit zu erreichen. In der Mitte des Semesters eröffnete mir E., dass sie mit
ihrer Abgabegruppe nicht zufrieden sei, da sich T. und D. als sehr unzuverlässig erwiesen. Sie wies mich nach Anfrage
an, vorerst nichts zu unternehmen. Einige Wochen später sprachen wir erneut über die Vorlesung und sie erzählte mir von
ihren Plänen, den Studiengang zu wechseln, da sie nicht mehr damit zufrieden war. Zu diesem Zeitpunkt waren bereits ein
Großteil der Übungsblätter abgegeben worden. Nachdem sie den Studiengang abgebrochen hatte, hörte die Abgabegruppe, welche
immer die selbe Handschrift aufwies, auf, regelmäßig Abgaben einzureichen. Diese Gruppe hat die Klausurzulassung nicht erreicht.

\subsection{Kontext zu meinen Bewertungen}
In meiner Korrektur habe ich unterschieden zwischen fachlichen bzw. inhaltlichen Fehlern und formalitätsbezogenen
Fehlern, die zwar semantisch korrekt waren, aber in einem falschen Format, wie zum Beispiel durch Komprimierung
unleserliche Dokumente oder falsche Dateibenennung. Einige Fehler ließen sich nicht klar Kategorisieren, dort habe
ich im Einzelfall entschieden. Bei Abgabegruppen, die knapp an der Grenze zum Bestehen lagen, habe ich auf deren Anfrage
kurz vor Ende des Semesters Punkte, die ich für Formalitätsfehler abgezogen hatte, wieder gutgeschrieben, um die Zulassung
nicht wegen für den späteren Berufsalltag irrelevanter Formalitäten zu gefährden oder das Studium gar um ein Semester zu
verzögern. Diese Option habe ich öffentlich im Tutorium angeboten und bei einer Abgabegruppe wie beschrieben durchgeführt.

\subsection{Klausurzulassungsanfrage}
Die Klausurzulassung wurden nach dem letzten Tutorium, aber klar vor der ersten Klausur eingestellt und klar kommuniziert.
Einige Wochen danach schrieb mir T. in Absprache mit D. eine Nachricht, in der sie E. als unzuverlässig
darstellten und damit begründen wollten, warum die Klausurzulassung von ihnen nicht erreicht werden konnte. Ich habe
mir deren Bepunktungen angeschaut und festgestellt, dass sie selbst in meinem entgegenkommendem Vorgehen, die Formalitätspunkte
wieder gutzuschreiben, die Zulassung um wenige Punkte verfehlt hätten. Außerdem wäre auf jeden Fall eine Kommunikation mit dem
zuständigen Professor notwendig gewesen, falls ich dem Ersuchen der Gruppe stattgegeben hätte.

\subsection{Mein Vorgehen}
Ich habe mich mit der Ansprechperson, die für die Organisation der Übungsblätter und die Koordination der Klausurzulassungen
zuständig gewesen ist, in Verbindung gesetzt und schilderte ihm die Situation. Ich gab an, dass ich mit E. befreundet war und
habe die Situation aus beiden Sichtweisen geschildert. Außerdem bat ich E. darum, eine Schriftprobe abzugeben.

Die Ansprechperson hat daraufhin entschieden, T. und D. keine Klausurzulassung zu erteilen, was ich in einer Antwortnachricht
an T. mitteilte.


\section{Reflexion des Falls}
Mein Fall besteht aus mehreren Aspekten, in denen ich Irritationen hatte, die jeweils durch meine Freundschaft mit E. entstanden
sind, aber auch meinen Gleichbehandlungsanspruch als Übungsleitender und mein Mitgefühl mit T. und D. beinhaltet haben.

\subsection{Allgemeine Beziehung zu E.}
Der erste Aspekt, der in diesem Fall eine Rolle spielt, war die Spannung, die zwischen dem Neutralitätsanspruch von mir und der
Universität und meiner Affektion zu E., die sich durch die freundschaftliche Beziehung zu ihr ergab, weil dadurch das Potential
für eine ungerecht positive Bewertung ihrer Abgaben, aber auch durch das Potential, wegen überdurchschnittlich hohem Feedback
von ihrer Seite aus die Tutorien in eine für sie hilfreiche Richtung zu lenken.

Neben den tatsächlichen Potentialen, galt es zusätzlich, die Wahrnehmung dieser Probleme bei den anderen Studierenden meiner Übungsgruppe nicht aufkommen zu lassen.

Für ersteres, das zu positive Bewerten, hatte ich bereits durch die Anonymisierung der Abgaben, die ich bekommen habe, gelöst habe.
Dadurch war das Potential deutlich kleiner, aber nicht komplett weg, da ich durch Handschriften doch einen, auch wenn anonymen,
Zusammenhang zwischen den jeweiligen Abgabegruppen über die jeweiligen Übungsblätter herstellen konnte. Eine optimalere Lösung
würde die Korrektur durch jeweils zufällige andere Tutoren bieten, was ich aber nicht angestoßen habe. Die Wahrnehmung dieses
Problems innerhalb meiner Übungsgruppe habe ich durch die Bekanntgabe der Anonymisierungsmethodik gelöst, was meine Studierenden
positiv wahrgenommen haben.

Letzteres habe ich während meiner Tätigkeit nicht aktiv betrachtet, ist aber durch die limitierte Kommunikation mit E. über das
Tutorium nicht aufgekommen. Die öffentliche Wahrnehmung von dem Thema ist auch nicht aufgekommen, ich habe jedem in meiner
Übungsgruppe die Möglichkeit gegeben, offenes Feedback zu geben. In Zukunft werde ich diese Offenheit Feedback gegenüber beibehalten,
da mir das Feedback neben der Möglichkeit zum Ausgleich potentieller Ungerechtigkeiten auch geholfen hat, mein Tutorium besser
zu machen und an meine Studierenden anzupassen.

Falls dieser Aspekt dieses Falls noch einmal auftreten sollte, werde ich nichts Zusätzliches
machen, um diese beiden Probleme zu verhindern, weil dieses Vorgehen sowohl meinen Gerechtigkeitssinn befriedigt hat als auch
durch bewusste Kommunikation meine Freundschaft zu E. nicht negativ beeinträchtigt hat.

\subsection{Umgang mit E.s Studiengangwechel}

\todo{ca. 2,5 Seiten}


\section{Fazit: Zusammenhang zum Seminar}
Zusammenhang zum Seminar, inwiefern ist Fallarbeit gut gelungen, Hilfreich etc.
ca. 1 Seite
\todo[inline]{Sollen wir Literatur nutzen? Wie? Referenzliteratur bei Systematisierungsdimensionen, Sitzung 10(nicht notwendig, ist aber potentiell hilfreich,
Fachbegriffe scheinen geil zu sein)}


\section{Eigenständigkeitserklärung}
Hiermit versichere ich, dass die vorliegende Praxisreflexion mit dem Titel
\textit{Theoriebasierte Praxisreflexion zum Berufsfeldpraktikum} von mir und
ohne Fremde Hilfe verfasst worden ist, dass keine anderen Quellen und Hilfsmittel
als die angegebenen benutzt worden sind und dass die Stellen der Arbeit, die
anderen Werken -- auch elektronischen Medien -- dem Wortlaut oder Sinn nach
entnommen wurden, auf jeden Fall unter Angabe der Quelle als Entlehnung kenntlich
gemacht worden sind. Mir ist bekannt, dass es sich bei einem Plagiat um eine Täuschung
handelt, die gemäß der Prüfungsordnung sanktioniert werden kann.

Ich erkläre hiermit, dass ich Kenntnis von einer zum Zweck der Plagiatskontrolle
vorzunehmenden Speicherung der Arbeit in einer Datenbank sowie von ihrem Abgleich mit
anderen Texten zwecks Auffindung von Übereinstimmungen habe.

Ich versichere, dass ich die vorliegende Arbeit oder Teile daraus nicht anderweitig als
Prüfungsarbeit eingereicht habe.

\vspace*{2cm}

\noindent
\begin{minipage}{0.48\textwidth}
  \begin{flushleft} \large
    \underline{\hspace{6cm}} \\
    {\footnotesize (Ort, Datum)}
  \end{flushleft}
\end{minipage}
\hfill
\begin{minipage}{0.48\textwidth}
  \begin{flushright} \large
    \underline{\hspace{6cm}} \\
    {\footnotesize (Unterschrift)}
  \end{flushright}
\end{minipage}\\
\end{document}