\documentclass[12pt]{article}
% Für spezialisierte Befehle etc.
% config.tex
\usepackage[
  a4paper,
  left=2.5cm,
  right=2.5cm,
  top=2.5cm,
  bottom=2cm
]{geometry}

\usepackage{tcolorbox}
\usepackage{amssymb}
\usepackage{amsmath}
\usepackage[ngerman]{babel}
\usepackage[T1]{fontenc}
\usepackage{pdfpages}
\usepackage[utf8]{inputenc}
\usepackage{tikz}
\usepackage{longtable}
\usepackage{array}
\usepackage{mathabx}
\usepackage{todonotes}
\usepackage[hidelinks]{hyperref}
\usepackage{setspace}
\usepackage{mathptmx}
\onehalfspacing
\usetikzlibrary{positioning, shapes.geometric, shapes.multipart, arrows, calc}
\tcbuselibrary{listings, skins, breakable}

\newcounter{block}[section]
\renewcommand{\theblock}{\thesection.\arabic{block}}

\usepackage{etoolbox}
\preto\section{\setcounter{block}{0}}

%Alle Sections beginnen auf einer neuen Seite
\let\oldsection\section
\renewcommand{\section}{\clearpage\oldsection}

\newcommand{\insertpdf}[2][0.8]{%
  \includepdf[
    pages=-,
    scale=#1,
    pagecommand={%
      \thispagestyle{plain}%
      \vspace*{0.5cm}%
      \centering\small\texttt{#2}%
    }
  ]{#2}%
}


\newtcolorbox{block}[1]{
  enhanced,
  breakable,
  colback=green!5!white,
  colframe=green!65!black,
  coltitle=black,
  fonttitle=\bfseries,
  title=\thesection.\number\numexpr\value{block}+1\relax\ #1,
  before upper={},
  after upper={\stepcounter{block}},
  boxrule=0.8pt,
  arc=0mm,
  left=1mm,
  right=1mm,
  top=1mm,
  bottom=1mm,
}

\newenvironment{beweis}[1][Beweis]{
  \par\noindent\textbf{#1.}\\ 
}{
    \hfill \framebox[1.5ex][c]{}
    \par\vspace{2em}
    \noindent
}

\newcolumntype{P}[1]{>{\centering\arraybackslash}p{#1}}

% Abgabe in 6 Wochen(vom 19.1. an, das ist der 2. März)

% Mail mit Betreff: Peter Minor, 514507, BFP-Seminar WiSe 25/26
% Gibt wohl keine Eingangsbestätigung

\begin{document}
% Titelseite
\thispagestyle{empty}
\noindent
Universität Münster \\
Fachbereich 06, Erziehungswissenschaft und Sozialwissenschaften \\
Wintersemester 2025/2026 \\
Vorbereitung und Begleitung des Berufsfeldpraktikums \\
Janusz Wilden \\
063591 \\

\vspace{2cm}

\begin{center}
\LARGE \textbf{Theoriebasierte Praxisreflexion zum Berufsfeldpraktikum}
\date{}
\end{center}

\vspace*{\fill}
\begin{flushright}
Peter Minor \\
514507\\
Boeselagerstraße 75b \\
48163 Münster \\
pminor@uni-muenster.de\\
Zweifach-Bachelor \\
Mathematik, Informatik \\
11. Fachsemester \\
Universität Münster \\
1. April 2025 - 30. September 2025\\
Abgabedatum \today \\
\end{flushright}
\newpage

% Inhaltsverzeichnis
\thispagestyle{empty}
\tableofcontents
\newpage


\section{Einleitung: Beschreibung der Praktikumstätigkeit}
Ich habe mein Praktikum an der Universität Münster als Tutor für die Vorlesung Informatik II absolviert,
die sich hauptsächlich an Studierende der Informatik, sowohl Hauptfachlich, als auch als Fach beim Zweifachbachelor
Informatik und an Wirtschaftsinformatikstudierende richtet. Sie behandelt weiterführend die
Grundkonzepte des objektorientierten Programierens und der Softwareentwicklung und -planung, insbesondere Rekursion,
Datenstrukturen und Komplexitätsanalyse von Algorithmen.

Ich habe einen Großteil meiner Tätigkeit im Homeoffice bzw. digital ausgeführt, die Zeit, die ich auf
dem Campus an der Einsteinstraße 64 verbrachte, hatte ich einen mir zugewiesenen Raum im Seminarraumzentrum am Orléans-Ring 12, den ich für die Zeit meines Tutoriums
frei nutzen durfte.

Mein Tätigkeitsbereich umfasste das Korrigieren von Abgaben der Übungsblätter, die die Studierenden im
Zuge der Vorbereitung auf die Klausur und im Rahmen einer Studienleistung einreichen mussten. Dazu waren
die Studierenden in Gruppen von 2 oder 3 Studierenden eingeteilt. Diese Abgaben erfolgten wöchentlich,
für die Korrektur war sowohl eine Musterlösung als auch ein Bewertungsschema vorgegeben. Außerdem zählte
das Halten eines Tutoriums zu meinem Aufgabenbereich, in dem ich die Vorlesungsinhalte der vergangengen
Woche und jeweils die Korrektur des vorangegangenen Übungsblattes und eine Vorbesprechung des aktuell zu
bearbeitenden Übungsblattes behandelte. Dabei sollte ich die Vorlesungsinhalte auf eine neue Art und an die jeweiligen
Kenntnisse der Studierenden, die ich aus der Qualität der Abgaben herauslesen sollte, angepasst darstellen und
Feedback zu den Abgaben geben, sowie die Probleme, die sich durch die Lernstandskontrollen in Form der Abgaben
gezeigt haben, ansprechen und bei der Lösungsfindung unterstützen. Dieses Feedback sollte zusätzlich zu der
Planung und Durchführung des Tutoriums auch einen Einfluss auf die Korrektur der Abgabe haben, was in Form von
Korrekturkommentaren und Bepunktungen erfolgen sollte.
Neben den Vorgegangenen, welche meine Hauptaufgaben darstellten, sollte ich den Studierenden auch organisatorisch
und beratend zur Seite stehen. Ich hatte eine feste Gruppe aus 16 Studierenden, die zu meiner Übungsgruppe
gehörten, also zu meinem Tutorium kamen, mich als ersten Ansprechpartner hatten und deren Übungszettelabgaben
ich beurteilte. Der letzte Teil meiner Arbeitszeit bestand aus einer Übungsleitendenbesprechung, in der alle
Tutoren sich versammelten und der Professor, der die Vorlesung hielt, uns über das aktuelle Geschehen in der
Vorlesung und die jeweils aktuellen Übungsblätter informierte. Ein großer Teil dieser Tätigkeiten ist
Eigenständig erfolgt.

Bis auf die Übungsleitendenbesprechung hatte ich keinen professionellen Kontakt zu meinen Mitübungsleitenden
oder dem Professorenteam. Meine Studierenden hatten die Möglichkeit, mich per Mail, Learnweb oder mithilfe
einer eigens für eine unkomplizierte Kommunikation eingerichtete Whatsapp-Gruppe zu kontaktieren. 


\section{Beschreibung des Fallbeispiels}
\subsection{Verhältnis zu E.}
In meiner Übungsgruppe gab es eine Studierende Person E., mit der ich bereits persönlich Kontakt hatte. Ich
ging regelmäßig mit ihr in die Mensa zum Essen oder traf sie im Kontext der gemeinsamen Freundesgruppe.
Unser freundschaftlicher Kontakt und die Begegnungen im universitären Kontext waren nur dadurch verschränkt, dass wir
im freundschaftlichen Kontext über die Vorlesung sprachen und sie mir Feedback zu meiner Tutoriumsgestaltung gab. Anders
herum war der einzige Einfluss, dass ich ihren Namen bereits kannte und wir nach den Tutorien gemeinsam essen
gingen oder kurz abglichen, ob wir zu einer Verabredung innerhalb der Freundesgruppe gehen würden.

\subsection{Abgabegruppe}
Am Anfang des Semesters werden die jeweiligen Abgabegruppen eingeteilt, in denen die Übungsblätter zusammen bearbeitet
und abgegeben werden. E. hat sich in der von mir dafür vorgesehenen Kennlernphase einer Gruppe zweier befreundeter
Studierenden, Person T. und Person D., angeschlossen. Ich anonymisierte die Abgaben, für mein persönliches
Fairnessempfinden, also wusste ich nicht, welche Personen welche Abgabe machten. Mir ist eine Abgabegruppe aufgefallen,
die immer diesselbe Handschrift hatte, was ich mir dadurch erklärte, dass eine Person der Abgabegruppe die Gemeinsamen
Lösungen aufschrieb, um eine bessere Lesbarkeit zu erreichen. In der Mitte des Semesters eröffnete mir E., dass sie mit
ihrer Abgabegruppe nicht zufrieden sei, da sich T. und D. als sehr unzuverlässig erwiesen. Sie wies mich nach Anfrage
an, vorerst nichts zu unternehmen. Einige Wochen später sprachen wir erneut über die Vorlesung und sie erzählte mir von
ihren Plänen, den Studiengang zu wechseln, da sie nicht mehr damit zufrieden war. Zu diesem Zeitpunkt waren bereits ein
Großteil der Übungsblätter abgegeben worden. Nachdem sie den Studiengang abgebrochen hatte, hörte die Abgabegruppe, welche
immer die selbe Handschrift aufwies, auf, regelmäßig Abgaben einzureichen. Diese Gruppe hat die Klausurzulassung nicht erreicht.

\subsection{Kontext zu meinen Bewertungen}
In meiner Korrektur habe ich unterschieden zwischen fachlichen bzw. inhaltlichen Fehlern und formalitätsbezogenen
Fehlern, die zwar semantisch korrekt waren, aber in einem falschen Format, wie zum Beispiel durch Komprimierung
unleserliche Dokumente oder falsche Dateibenennung. Einige Fehler ließen sich nicht klar Kategorisieren, dort habe
ich im Einzelfall entschieden. Bei Abgabegruppen, die knapp an der Grenze zum Bestehen lagen, habe ich auf deren Anfrage
kurz vor Ende des Semesters Punkte, die ich für Formalitätsfehler abgezogen hatte, wieder gutgeschrieben, um die Zulassung
nicht wegen für den späteren Berufsalltag irrelevanter Formalitäten zu gefährden oder das Studium gar um ein Semester zu
verzögern.\\
Diese Option habe ich öffentlich im Tutorium angeboten und bei einer Abgabegruppe wie beschrieben durchgeführt.

\subsection{Klausurzulassungsanfrage}
Die Klausurzulassung wurden nach dem letzten Tutorium, aber klar vor der ersten Klausur eingestellt und klar kommuniziert.
Einige Wochen danach schrieb mir T. in Absprache mit D. eine Nachricht, in der sie E. als unzuverlässig
darstellten und damit begründen wollten, warum die Klausurzulassung von ihnen nicht erreicht werden konnte. Ich habe
mir deren Bepunktungen angeschaut und festgestellt, dass sie selbst in meinem entgegenkommendem Vorgehen, die Formalitätspunkte
wieder gutzuschreiben, die Zulassung um wenige Punkte verfehlt hätten. Außerdem wäre auf jeden Fall eine Kommunikation mit dem
zuständigen Professor notwendig gewesen, falls ich dem Ersuchen der Gruppe stattgegeben hätte.

\subsection{Mein Vorgehen}
Ich habe mich mit der Ansprechperson, die für die Organisation der Übungsblätter und die Koordination der Klausurzulassungen
zuständig gewesen ist, in Verbindung gesetzt und schilderte ihm die Situation. Ich gab an, dass ich mit E. befreundet war und
habe die Situation aus beiden Sichtweisen geschildert. Außerdem bat ich E. darum, eine Schriftprobe abzugeben.

Die Ansprechperson hat daraufhin entschieden, T. und D. keine Klausurzulassung zu erteilen, was ich in einer Antwortnachricht
an T. mitteilte.


\section{Reflexion des Falls}
Mein Fall besteht aus mehreren Aspekten, deren zu Grunde liegende Irrationen ich betrachte. Diese waren hauptsächlich durch
meine freundschaftliche Beziehung mit E., sowie auch durch meinen Gleichbehandlungsanspruch als Übungsleitender und mein
Verständnis von T. und D.s Situation. Ich betrachte im Nachhinein drei Situationen, beschreibe, wie meine Entscheidungsprozesse
ausgesehen haben und reflektiere diese im Anschluss. Sie stellen für mich verschiedene, zeitlich getrennte Situationen, aber
durch die wechselseitige Abhängigkeit verschmelzen sie im Sinne einer Fallarbeit zu einem Fall.

\subsection{Allgemeine Beziehung zu E.}
Der erste Aspekt, der in diesem Fall eine Rolle spielt, war die Spannung, die zwischen dem Neutralitätsanspruch, sowohl von mir als auch
von der Universität, und meiner Affektion zu E., die sich durch die freundschaftliche Beziehung zu ihr ergab, weil dadurch das Potential
für eine ungerecht positive Bewertung ihrer Abgaben, aber auch durch das Potential, auf Grund überdurchschnittlich häufigem Feedback
von ihrer Perspektive aus, die Tutorien in eine für sie hilfreiche Richtung zu lenken, gesteigert sein könnte.

Neben diesen Potentialen galt es zusätzlich, die Wahrnehmung der Spannung bei den anderen Studierenden meiner Übungsgruppe nicht
aufkommen zu lassen, weil diese das Vertrauen in meine Neutralität verletzen könnte.

Das zu positive Bewerten hatte ich bereits durch die Anonymisierung der Abgaben gelöst.
Dadurch war das Potential deutlich kleiner, aber nicht komplett anulliert, da ich durch Handschriften doch einen, auch wenn anonymen,
Zusammenhang zwischen den jeweiligen Abgabegruppen über die jeweiligen Übungsblätter hinweg herstellen konnte. Eine optimalere Lösung
würde die Korrektur durch jeweils andere Tutoren darstellen, was ich aber nicht angestoßen habe, weil das einen großen Eingriff in
das Gesamtsystem darstellen würde. Das Problem der Wahrnehmung dieser
Spannung innerhalb meiner Übungsgruppe habe ich durch die Bekanntgabe der Anonymisierungsmethodik gelöst, was meine Studierenden
positiv wahrgenommen haben.

Die Feedbackproblematik habe ich während meiner Tätigkeit nicht wahrgenommen, sie ist durch die limitierte Kommunikation mit E. über das
Tutorium auch nicht aufgetreten. Die Spannung ist in der Gruppe nicht wahrgenommen worden, ich habe allen Teilnehmenden meiner
Übungsgruppe die Möglichkeit gegeben Feedback zu geben. In Zukunft werde ich diese Offenheit beibehalten,
da mir das Feedback neben der Möglichkeit zum Ausgleich potentieller Ungerechtigkeiten auch geholfen hat, mein Tutorium zu verbessern
und an meine Studierenden anzupassen.

Falls solche Situationen erneut auftreten, werde ich dasselbe
machen, um diese beiden Probleme zu umgehen, weil dieses Vorgehen sowohl meinem Gerechtigkeitssinn entsprach als auch
durch bewusste Kommunikation meine Freundschaft zu E. nicht negativ beeinträchtigte.

\subsection{Studiengangwechel von E.}
Der zweite Aspekt, der in diesem Fall als Irritation betrachtet wird, ist E.s Studiengangwechsel. In der Situation
standen sich das Prinzip der minimalen Hilfestellung und mein Zweifel, ob die Studierenden T. und D. durch E.s Verlassen der Gruppe
eine weitergehende Hilfestellung bedürfen könnten, gegenüber.

Im Allgemeinen habe ich für das Semester das Prinzip der minimalen Hilfestellung verfolgt.
Das Prinzip der minimalen Hilfestellung beschreibt, durch die kleinste sinnvolle Unterstützung zu erreichen,
dass die jeweiligen Ziele der Lernenden erreicht werden, aber ihnen gleichzeitig möglichst viel Raum zu geben, Probleme
selbstständig zu lösen und dadurch zu wachsen.

Dieses Prinzip auch in die organisatorischen Teile des Studiums, wie den Umgang mit Problemen in der Abgabegruppe, auszuweiten,
hätte den Vorteil, dass die Studierenden die Möglichkeit hätten, auch in diesen Bereichen zu wachsen. Dagegen sprach in dieser
Situation, dass die Studierenden vielleicht keine Erfahrungen mit solchen Problemstellungen hätten, was ohne Hilfestellung bei
organisatorischen Aufgaben verheerende Auswirkungen auf den weiteren Studienverlauf haben könnte.

Ich entschied mich am Anfang des Semesters für eine Implementierung des Prinzips der minimalen Hilfestellung, aber mit Ausschluss der organisatorischen Themen, die im
Studium neuartig sein könnten, wie die Prüfungsanmeldungen und das Organisieren von Abgaben in Kleingruppen, und potentiell das
Voranschreiten des Studiums verhindern könnten.

Das habe ich erreicht, indem ich in den Tutorien immer
nur die Stellen der Übungsblätter angesprochen habe, die in den Abgaben besonders schlecht gelaufen sind und diese jeweils
durch ähnliche Aufgaben und unterstützende Hilfestellungen während der Tutorien üben ließ. Dadurch hatten die Studierenden
genau an den Stellen, die ihnen schwergefallen sind, eine Unterstützungs- und Übungsmöglichkeit.

In meinen Anwendungsbereich des Prinzips fällt die beschriebene Situation insofern hinein, dass durch das Fehlen
eines Gruppenmitglieds potentiell verhindert werden könnte, dass die Klausurzulassung erreicht wird. Ob eine neuartige Problemstellung
für T. und D. bestand, ist nicht mit Sicherheit zu sagen, meiner Einschätzung nach war dies aber nicht der Fall.


Das hatte zur Konsequenz, dass ich nicht eingegriffen habe, so lange nicht um Hilfe gebeten wurde.

Als dieser Aspekt dieses Falls in einem späteren Tutorium noch einmal aufgetreten ist, habe ich diese
Einschätzung nicht noch einmal so getroffen, da sie in dem beschriebenen Fall zum nicht-erreichen der Klausurzulassung beigetragen
hat, also die Vermutung naheliegt, dass die Studierenden T. und D. nicht genügend Erfahrung hatten, um mit der Situation adäquat umzugehen.

\subsection{Anfrage der Abgabegruppe}
Der dritte Aspekt, der in diesem Fall als Krise betrachtet wird, ist T.s und D.s Anfrage. Durch diese Situation hat sich mir gezeigt,
dass mein Umgang mit dem zweiten Aspekt verbesserungswürdig war. Hier wird eine der Antinomien im Lehrberuf klar: Ich musste zwar in
der Situation sehr spontan entscheiden, wie ich mit der Situation umgehen soll, musste mich aber gegenüber den Studierenden implizit
erklären, warum von meiner Seite keine Unterstützung gekommen war, als die Abgaben immer unregelmäßiger geworden sind, was ein Beispiel 
für die Begründungsantinomie ist. Diesselbe Erklärungsnot hätte ich auch gegenüber dem Professorenteam gehabt, falls herauskommen sollte,
dass ich über E.s Wechsel informiert gewesen bin.

Dadurch ergab sich ein neues Dilemma: Entweder könnte ich die Entscheidung treffen, die Information zurückzuhalten und mich somit
einer Begründungsnot zu entziehen, in einer Situation, in der ich zu der Zeit sehr unsicher war und keine Begründung in der Hand
hatte, oder mich diesem Widerspruch stellen und diesen Aufarbeiten, um dann eine begründete Informationsweiterleitung zu betreiben.
Dann müsste ich mich einer Bewertung meiner Entscheidungen stellen und potentielle Konsequenzen über mich ergehen lassen. Für
letzteres habe ich mich in dieser Situation entschieden.

Falls dieser Aspekt dieses Falls noch einmal auftreten sollte, werde ich dasselbe tun, die Aufarbeitung der Situation hat mir geholfen,
ein besseres Verständnis für meine Erwartungen an mich als didaktische Kraft zu entwickeln und hat meine weiteren Handlungen in
positiver Weise beeinflusst.


\section{Fazit: Zusammenhang zum Seminar}
Der dargestellte Fall zeigt exemplarisch, wie komplex und vielschichtig das professionelle Handeln im Lehrkontext sein kann, vor allem
durch Spannungen zwischen verschiedenen Aufgaben und Verantwortungen, die eine Lehrperson jeweils gegenüber verschiedenen Akteuren,
wie die lernenden Personen, dem Arbeitgeber oder Eltern hat.

Die Auseinandersetzung mit diesem Fall nach den im Seminar besprochenen Methoden hat mir geholfen, diese Spannungen nicht nur rückblickend
zu beschreiben, sondern sowohl diese als auch meinen Umgang mit ihnen systematisch zu reflektieren und in einen theoretischen Zusammenhang
einzuordnen. Zentral war dabei die Erfahrung, dass viele Herausforderungen im Lehrberuf nicht durch eindeutige Regeln oder einfache
Lösungen aufzulösen sind, sondern als Antinomien bestehen bleiben, was es einfacher macht, mit nicht perfekten Lösungen zufrieden zu sein.
In meinem Fall zeigte sich diese Inperfektion besonders im Spannungsfeld zwischen
Nähe und Distanz, zwischen Gleichbehandlungsanspruch und individueller Beziehung sowie zwischen minimaler Hilfestellung und notwendigem
Eingreifen. Die Fallarbeit im Seminar hat mir geholfen, diese Widersprüche nicht als persönliches Scheitern oder Fehlverhalten zu
interpretieren, sondern als strukturelle Merkmale pädagogischen Handelns, die reflektiert ausgehalten und begründet bearbeitet werden
müssen. Als besonders hilfreich habe ich die methodische Struktur der Fallarbeit empfunden. Durch die detaillierte Rekonstruktion des
Geschehens wurde mir teilweise erst im Nachhinein bewusst, an welchen Stellen implizite Annahmen und persönliche Überzeugungen mein Handeln
geleitet haben. Gerade die Auseinandersetzung mit Alternativentscheidungen und deren möglichen Konsequenzen hat meinen Blick dafür geschärft,
dass pädagogische Entscheidungen immer vorläufig sind und sich erst im Verlauf als tragfähig oder problematisch erweisen. Die Analyse des
zweiten Aspekts meines Falls hat mir deutlich gemacht, dass didaktische Prinzipien wie das der minimalen Hilfestellung zwar eine sinnvolle
Orientierung bieten, jedoch kontextsensitiv angewendet und gegebenenfalls angepasst werden müssen.

Darüber hinaus hat das Seminar mein Verständnis von professioneller Verantwortung erweitert. Insbesondere die Reflexion der Entscheidung,
Informationen transparent weiterzugeben und mich der Begründungsnot zu stellen, hat mir verdeutlicht, dass professionelles Handeln nicht nur
im konkreten Unterstützen von Lernenden besteht, sondern auch in der Fähigkeit, eigene Entscheidungen offen zu reflektieren und zu vertreten.
Die Fallarbeit hat mir gezeigt, dass gerade diese Bereitschaft zur Selbstkritik und Aufarbeitung eine zentrale Voraussetzung für professionelles
Wachstum darstellt. Insgesamt hat mir die Arbeit am eigenen Fall deutlich gemacht, dass Fallarbeit ein sehr geeignetes Instrument ist, um die
theoretischen Inhalte des Seminars mit der eigenen Praxis zu verknüpfen. Sie ermöglicht es, abstrakte Konzepte wie Antinomien, Neutralität oder
pädagogische Verantwortung anhand konkreter Erfahrungen zu durchdringen und nachhaltig zu verinnerlichen. Für meine zukünftige Tätigkeit als
Lehrkraft nehme ich aus dem Seminar mit, dass Unsicherheit und Irritation nicht vermieden werden müssen, sondern wertvolle Ausgangspunkte für
Reflexion und Weiterentwicklung darstellen können.


\section{Eigenständigkeitserklärung}
Hiermit versichere ich, dass die vorliegende Praxisreflexion mit dem Titel
\textit{Theoriebasierte Praxisreflexion zum Berufsfeldpraktikum} von mir und
ohne Fremde Hilfe verfasst worden ist, dass keine anderen Quellen und Hilfsmittel
als die angegebenen benutzt worden sind und dass die Stellen der Arbeit, die
anderen Werken -- auch elektronischen Medien -- dem Wortlaut oder Sinn nach
entnommen wurden, auf jeden Fall unter Angabe der Quelle als Entlehnung kenntlich
gemacht worden sind. Mir ist bekannt, dass es sich bei einem Plagiat um eine Täuschung
handelt, die gemäß der Prüfungsordnung sanktioniert werden kann.

Ich erkläre hiermit, dass ich Kenntnis von einer zum Zweck der Plagiatskontrolle
vorzunehmenden Speicherung der Arbeit in einer Datenbank sowie von ihrem Abgleich mit
anderen Texten zwecks Auffindung von Übereinstimmungen habe.

Ich versichere, dass ich die vorliegende Arbeit oder Teile daraus nicht anderweitig als
Prüfungsarbeit eingereicht habe.

\vspace*{2cm}

\noindent
\begin{minipage}{0.48\textwidth}
  \begin{flushleft} \large
    \underline{\hspace{6cm}} \\
    {\footnotesize (Ort, Datum)}
  \end{flushleft}
\end{minipage}
\hfill
\begin{minipage}{0.48\textwidth}
  \begin{flushright} \large
    \underline{\hspace{6cm}} \\
    {\footnotesize (Unterschrift)}
  \end{flushright}
\end{minipage}\\
\end{document}